\section{Zielsetzung}
\label{sec:Zielsetzung}

Ziel des Versuches ist es die Temperaturabhängigkeit der Molwärme von Kupfer zu messen. Diese wird mit den Vorhersagen 
vom klassischen Dulong-Petit-Gesetz sowie mit dem Einstein- und dem Debye-Gesetz zur Molwärme verglichen. Zusätzlich wird ein Wert für die Debye-Temperatur $\theta_D$ bestimmt und mit dem Theoriewert verglichen.


\section{Theorie}
\label{sec:Theorie}

\subsection{Dulong-Petit-Gesetz}
\label{sec:dulongpetit}

Atome in einem Festkörper sind durch Gitterkräfte an festen Stellen gebunden. Sie besitzen jedoch drei Freiheitsgrade, in die sie schwingen können. 
Im Mittel ist deren kinetische Energie gleich der potentiellen Energie. Die Energie pro Freiheitsgrad im klassischen Modell ist gegeben durch 
$E = \frac{1}{2}k_B T$. Dabei ist $k_B$ die Boltzsmann-Konstante und $T$ die Temperatur. Die mittlere Energie eines Atoms in drei Dimensionen beträgt also 
\begin{align*}
    E = E_{\text{kin}} + E_{\text{pot}} = \frac{3}{2}k_B T + \frac{3}{2}k_B T = 3k_B T.
\end{align*}
Bei einem Mol in einem Festkörper ist die innere Energie 
\begin{align*}
    U = 3 \cdot N_L\cdot k_B \cdot T.
\end{align*}

Hierbei steht $N_L$ für die Loschmidt-Zahl. Somit ist die klassische spezifische Molwärme gegeben durch eine Konstante:
\begin{align}
    C_V = \( \frac{\partial E}{\partial T} \)_V = 3 N_L k_B = 3 R
\end{align}
In der klassischen Betrachtung ist die Molwärme also sowohl Material als auch Temperaturunabhängig. Dieses Eigenschaft nennt man das Dulong-Petit-Gesetz.
\subsection{Einstein-Gesetz}
\label{sec:einstein}
%Rest kommt noch
\subsection{Debye-Gesetz}
\label{sec:debye}