\section{Zielsetzung}
\label{sec:Zielsetzung}

Ziel des Versuches ist es die Temperaturabhängigkeit der Molwärme von Kupfer zu messen. Diese wird mit den Vorhersagen 
vom klassischen Dulong-Petit-Gesetz sowie mit dem Einstein- und dem Debye-Gesetz zur Molwärme verglichen. Zusätzlich wird ein Wert für die Debye-Temperatur $\theta_D$ bestimmt und mit dem Theoriewert verglichen.


\section{Theorie}
\label{sec:Theorie}
Die Molwärme ist bei Festkörpern ein Maß für die Fähigkeit Wärme aufzunehmen und abzugeben. Sie wird definiert als die Energie, die in Form von Wärme zugeführt werden muss, 
um den Festkörper um $1\,\si{°C}$ zu erwärmen: 
\begin{align*}
    C = \frac{\delta Q}{\delta T}.
\end{align*}
Die spezifische Wärmekapazität ist die Molwärme pro Masse.\\
Um die Molwärme zu bestimmen werden verschiedene Modelle zur Näherung verwendet. Diese sind zum einen das klassische Dulong-Petit-Gesetz und die quantenmechanischen 
Modelle von Einstein und Debye. 
\subsection{Dulong-Petit-Gesetz}
\label{sec:dulongpetit}

Atome in einem Festkörper sind durch Gitterkräfte an festen Stellen gebunden. Sie besitzen jedoch drei Freiheitsgrade, in die sie schwingen können. 
Im Mittel ist deren kinetische Energie gleich der potentiellen Energie. Die Energie pro Freiheitsgrad im klassischen Modell ist gegeben durch 
$E = \frac{1}{2}k_B T$. Dabei ist $k_B$ die Boltzsmann-Konstante und $T$ die Temperatur. Die mittlere Energie eines Atoms in drei Dimensionen beträgt also 
\begin{equation}
    E = E_{\text{kin}} + E_{\text{pot}} = \frac{3}{2}k_B T + \frac{3}{2}k_B T = 3k_B T.
\end{equation}
Bei einem Mol in einem Festkörper ist die innere Energie 
\begin{equation}
    U = 3 \cdot N_L\cdot k_B \cdot T.
\end{equation}

Hierbei steht $N_L$ für die Loschmidt-Zahl. Somit ist die klassische spezifische Molwärme gegeben durch eine Konstante:
\begin{equation}
    C_V = \frac{\partial E}{\partial T} \bigg\vert_V = 3 N_L k_B = 3 R
\end{equation}
In der klassischen Betrachtung ist die Molwärme also sowohl Material als auch Temperaturunabhängig. Diese Eigenschaft nennt man das Dulong-Petit-Gesetz.
\subsection{Einstein-Gesetz}
\label{sec:einstein}

Das Modell von Albert Einstein befasst sich mit einer quantenmechanischen Näherung zur Bestimmung der Molwärme. In dem Einstein-Modell ist die Annahme, 
dass alle Gitterschwingungen mit der gleichen Frequenz $\omega$ schwingen. Diese Gitterschwingungen werden des Weiteren auch gequantelt betrachtet und besitzen demnach ein 
Vielfaches von $\hbar \omega$ als Energie. Diese gequantelten Gitterschwingungen werden auch Phononen genannt. \\

Die Energien sind im Festkörper boltzmann-verteilt. Dadurch kann die Wahrscheinlichkeit, dass ein Phonon bei thermischem Gleichgewicht bei der Temperatur $T$ die Energie $n \hbar \omega$ besitzt, 
einfach mit der Formel
\begin{align}
    \label{eqn:wn}
    W(n) = \exp{\left( - \frac{n \hbar \omega}{k_B T} \right)}
\end{align}
berechnet werden. \\
Der Erwartungswert $X$ der Energie wird berechnet, indem die über alle $n$ zwischen $0$ und $\infty$ mit den entsprechenden $W(n)$ Koeffizienten summiert wird. 
Diese Rechnung ergibt
\begin{align}
    X_{\text{Einstein}} = \frac{\hbar \omega}{\exp \left( \frac{\hbar \omega}{k_B T}\right) -1} .
\end{align} 
Die mittlere Energie im Einstein-Modell beträgt also 
\begin{align}
    \langle u \rangle_{\text{Einstein}} = \frac{X}{Z} = \frac{\hbar \omega }{\exp \left( \frac{\hbar \omega}{k_B T}\right)}.
\end{align}
Die Zustandssumme $Z$ gibt dabei die Summe der gesamten Zustände an. Abschließend ergibt sich die Molwärme zu 
\begin{align}
    \label{eqn:moleinstein}
    (C_V)_{\text{Einstein}} = 3 R \left( \frac{\hbar \omega}{k_B T}\right)^2 \frac{\exp \left(\frac{\hbar \omega}{k_B T}\right)}{\left(\exp \left(\frac{\hbar \omega}{k_B T}\right) -1 \right)^2}.
\end{align}
Es ist festzustellen, dass die Molwärme im Einstein-Modell gegen den klassischen Wert $3 R$ bei hohen Temperaturen konvergiert. Bei niedrigen Temperaturen also $T \rightarrow 0$ läuft die Molwärme jedoch gegen $0$.
\subsection{Debye-Gesetz}
\label{sec:debye}

Im Vergleich zum Einstein-Modell, wird im debyeschem Modell angenommen, dass Phononen bis zu einer Grenzfrequenz $\omega_D$ eine Verteilung von Frequenzen besitzen können. 
Eine Grenzfrequenz existiert, weil ein Festkörper eine endlich große Dimension und somit auch nur endlich viele Eigenschwingungen hat. Diese Grenzfrequenz wird Debye-Frequenz genannt. Es wird auch angenommen, 
dass bis zu der Grenzfrequenz eine lineare Dispersionsrelation, also $\omega = \nu_s k$, gegeben ist. 
Die Anzahl der Eigenschwingungen beträgt $3 N_L$ wobei $N_L$ die Loschmidt-Konstante darstellt. Die Loschmidt-Konstante gibt die Anzahl $N$ der Moleküle pro Volumen $V$ eines idealen Gases 
bei einer Temperatur von $T = 273, \! 15\, \si{\kelvin}$ und einem Druck von $p = 101, \! 325 \, \si{\kilo\pascal}$ an also:
\begin{align}
    N_L = \frac{N}{V} = \frac{p}{k_B T}.
\end{align}
Wird die Frequenzverteilung $Z(\omega)$ bis zu der Grenzfrequenz $\omega_D$ integriert, so sollte also $3 N_L$ rauskommen
\begin{align}
    \label{eqn:Z_integral}
    \int_{\, 0}^{\omega_D} Z(\omega) d\omega = 3 N_L .
\end{align}
Um Einblicke aus diesem Integral zu erhalten, ist es nützlich die Zustandsdichte $Z(\omega)$ im $k$-Raum zu schreiben. Das geht über die Umformung 
\begin{align}
    Z(\omega) d\omega &= Z(k) d k\\
    \Leftrightarrow \, \, \, \, \, \, \, \, \, \, \, \, Z(\omega) &= Z(k) \frac{d k}{d\omega}
\end{align}
Die Zustandsdichte im $k$-Raum ist im dreidimensionalem Fall immer 
\begin{align}
    Z(k) = \frac{L^3}{\left(2 \pi\right)^3} \frac{4 \pi k^3}{3} = \frac{V}{8 \pi^3} \frac{4 \pi k^3}{3}
\end{align}
also ist die Zustandsdichte im Frequenzraum
\begin{align}
    \label{eqn:Z}
    Z(\omega) d \omega &= \frac{L^3}{2 \pi^2} \omega^2 \left(\frac{1}{\nu^3_{ges}} \right) d\omega= \frac{L^3}{2 \pi^2} \omega^2 \left(\frac{1}{\nu^3_l} + \frac{2}{\nu^3_{tr}}\right) d\omega \\
    Z(\omega) d \omega &= \frac{9 N_L}{\omega_D^3} \omega^2 d \omega.
\end{align}

Die Gesamtenergie $U$ ergibt sich dann durch das Integrieren über alle Frequenzen

\begin{align}
    \label{eqn:udebye}
    U &= \int_{\, 0}^{\omega_D} \frac{Z(\omega) \, \hbar \omega}{\exp \left( \frac{\hbar \omega}{k_B T} - 1 \right)} d\omega.
\end{align}

Abschließend ergibt sich die Molwärme zu 

\begin{align}
    C_{V, \text{Debye}} &= \frac{d}{d T} \frac{9 N_L}{\omega_D^3} \int_{\, 0}^{\omega_D} \frac{\hbar \omega^3}{\exp \left(\frac{\hbar \omega}{k_B T} -1 \right)} d \omega.
\end{align}

Damit lässt sich die innere Energie wie folgt berechnen:
\begin{align}
    U =  \int_{\, 0}^{\omega_D} Z(\omega) \langle n(\omega)\rangle \hbar \omega d \omega=  \int_0^{\omega_D} \frac{Z(\omega) \hbar \omega d \omega}{\left(\exp \left(\frac{\hbar \omega}{k_B T}\right) -1 \right)}.
\end{align}
wobei $\langle n(\omega)\rangle$ die Bose-Einstein-Statistik angibt. Die Ableitung nach $T$ der inneren Energie ergibt die Lösung für die Molwärme. Die Molwärme im Debye-Modell hat die Form
\begin{align}
    C_{V,\text{Debye}}= \frac{d}{d T}\frac{9 N_L}{\omega_D^3} \int_{\, 0}^{\omega_D} \frac{\omega^3 d\omega}{\left(\exp \left(\frac{\hbar \omega}{k_B T}\right) -1 \right)}
\end{align}
Durch Einsetzen der Substitution 
\begin{align}
    \label{eqn:debyesubst}
    x = \frac{\hbar \omega}{k_B T} \\  \frac{\theta_D}{T} = \frac{\hbar \omega_D}{ k_B T}
\end{align}
wird die Molwärme zu
\begin{align}
    C_{V,\text{Debye}} = 9 R \left(\frac{\theta_D}{T}\right)^3 \int_0^{\frac{\theta_D}{T}} \frac{x^4 \exp \left(x\right)}{ \left(\exp(x) -1\right)^2}.
\end{align}
Das $\theta_D$ wird die Debye-Temperatur genannt. Die Debye-Temperatur $\theta_D$ ist ein materialabhängiger Parameter. Sie gibt im Debye Modell die Temepratur an, bei der alle Zustände, die möglich sind, besetzt werden. Die Molwärme im Debye-Modell bei großen Temperaturen also $x<<1$ konvergiert gegen $3 R$, die klassischen Näherung. Bei kleinen Temperaturen ist die Molwärme 
jedoch proportional zu $T^3$. 
\subsection{Weitere Anmerkungen}
Es ist wichtig zu erwähnen, dass die Leitungselektronen im Festkörper ebenfalls einen Beitrag zur Molwärme leisten. Vor allem bei niedrigen Temperaturen spielt der Einfluss der Leitungselektronen eine größere Rolle. 
Die Elektronen sind nach der Fermi-Dirac-Statistik verteilt. Daraus folgt, dass der Einfluss auf die Molwärme der Leitungselektronen proportional zu $T$ sein muss. \\
