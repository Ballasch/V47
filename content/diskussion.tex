\section{Diskussion}
\label{sec:Diskussion}
Zunächst wird durch die Abbildungen \ref{fig:C_V} sofort ersichtlich, dass dieses Experiment von großen Fehlern besetzt ist. 
Es gibt nur wenige Zusammenhang zwischen den Messwerten und der theoretischen Vorhersage. Des Weiteren sind 
auch große Diskrepanzen zwischen den Messwerten zu erkennen, was auf einen hohen statistischen Fehler hinweist. In den Abbildungen 
\ref{fig:Theta_C_V} und \ref{fig:Theta_T} wird ebenfalls deutlich, dass die hier bestimmte Debye-Temperatur keinen ersichtlichen Trend aufweist 
und somit auch von hohen Fehlern betroffen ist. \\
Die Messung zu der Molwärme zeigt trotzdem klar, wie im Schnitt bei hohen Temperaturen die klassisch erwartete Molwärme von $3 R$ angenähert wird. 
Es ist auch ein Absinken der Molwärme bei kleinen Temperaturen zu erkennen. Aufgrund der hohen Fehler ist jedoch nicht zu erkennen, dass das Debye-Modell eine bessere Näherung 
bietet als das Einstein-Modell.  \\

Trotz hohen statistischen Messfehlern, liegt die gemessene Debye-Temperatur nah an der Theorieerwartung:
\begin{align*}
    \theta_{D,\text{gemessen}} &= (329 \pm 19) \si{\kelvin}\\
    \theta_{D,\text{theoretisch}} &= 332,102 \si{\kelvin}
\end{align*}
Die prozentuale Abweichung des gemessen Wertes von der theoretischen Berechnung beträgt also
\begin{align*}
    \Delta \theta_D = (0.94\pm 5) \si{\%}.
\end{align*}
\\ 
Der Großteil der Fehler entstehen aufgrund der mangelhaften Messmethode in diesem Experiment. Da die Temperatur im Dewar-Gefäß mit flüssigem Stickstoff 
gering gehalten wird und dieses per Hand ständig nachgekippt werden musste, konnte keine kontinuierlich steigende Temperatur erzielt werden. 
Das hat dazu geführt, dass die Temperaturunterschiede zwischen dem Gefäß und der Probe zum Teil sehr unterschiedlich waren. 
Dieses Temperaturproblem trägt höchtwahrscheinlich den größten Beitrag zum Fehler. 
Außerdem ist der Aufbau nicht gut abgeschirmt. Das Dewar-Gefäß war oben offen und somit anfällig gegen äußere Einflüsse auf die Temperatur. 
Es ist ebenfalls wichtig zu erwähnen, dass die Zeit nur ungenau gemessen werden konnte, da das auftragen der Messwerte ebenfalls bis zu ca. eine Minute gedauert hat. 
