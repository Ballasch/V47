\section{Diskussion}
\label{sec:Diskussion}
Zunächst wird durch die Abbildungen \ref{fig:C_V} sofort ersichtlich, dass dieses Experiment von großen Fehlern besetzt ist. 
Es gibt nur wenige Zusammenhang zwischen den Messwerten und der theoretischen Vorhersage. Des Weiteren sind 
auch große Diskrepanzen zwischen den Messwerten zu erkennen, was auf einen hohen statistischen Fehler hinweist.. \\
Die Messung zu der Molwärme zeigt trotzdem klar, wie im Schnitt bei hohen Temperaturen die klassisch erwartete Molwärme von $3 R$ angenähert wird. 
Es ist auch ein Absinken der Molwärme bei kleinen Temperaturen zu erkennen.\\
 In der Abbildung \ref{fig:Theta_T} lässt sich erkennen, dass $\theta_D$ mit der Temperatur steigt. Dies lässt sich mit der Volumenabhängigkeit von V erklären. Die zwei unterschiedlichen Steigungen sind hängen demnach mit der longitudinalen und transversalen Schallgeschwindigkeit zusammen. 

Die gemessene Debye-Temperatur nah an der Theorieerwartung:
\begin{align*}
    \theta_{D,\text{gemessen}} &= (323.43 \pm 24.25) \si{\kelvin}\\
    \theta_{D,\text{theoretisch}} &= 332.102 \si{\kelvin}
\end{align*}
Die prozentuale Abweichung des gemessen Wertes von der theoretischen Berechnung beträgt also
\begin{align*}
    \Delta \theta_D =\frac{\theta_{D,\text{theoretisch}}}{\theta_{D,\text{gemessen, min}}} -1 =  2,7 \,\si{\%}
\end{align*}
Man sollte jedoch hierbei anmerken, dass der gemessene Wert stark fehlerbehaftet ist und diese geringe Abweichung somit großenteils aus Glück entstanden ist. \\\\
Der Großteil der Fehler entstehen aufgrund der ungenauen Messpräzision per Hand in diesem Experiment. Da die Temperatur im Dewar-Gefäß mit flüssigem Stickstoff 
gering gehalten wird und dieses per Hand ständig nachgekippt werden musste, konnte keine kontinuierlich steigende Temperatur erzielt werden. 
Das hat dazu geführt, dass die Temperaturunterschiede zwischen dem Gefäß und der Probe zum Teil sehr unterschiedlich waren. 
Bei der Durchführung gab es auch Probleme mit der Stoppuhr, wodurch große Fehler in der Zeitmessung in den ersten Werten entstanden sind. Des Weiteren wurde spät bemerkt, dass der Generator eine sehr ungenaue Spannung angegeben hat. Die Messung der Spannung wurde an späteren Datenpunkten genauer, da ein weiteres Spannungsmessgerät angeschlossen wurde.   
Außerdem ist der Aufbau nicht gut abgeschirmt. Das Dewar-Gefäß war oben offen und somit anfällig gegen äußere Einflüsse auf die Temperatur. 
